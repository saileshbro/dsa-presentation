\documentclass[aspectratio=169]{beamer}
% For best font rendering with XeLaTeX/LuaLaTeX
\usepackage{fontspec}
\usepackage{unicode-math}
\usepackage{tikz}
\usepackage{colortbl}
\usepackage{amsmath}
\usepackage{amssymb}
\usepackage{graphicx}
\usepackage{hyperref}
\usepackage{algorithm}
\usepackage{algpseudocode}
\usepackage{booktabs}
\usepackage{multirow}
\usepackage{xcolor}
\usepackage{listings}
\usepackage{biblatex}

% TikZ libraries
\usetikzlibrary{arrows,automata,positioning,shapes,calc,decorations.pathmorphing,patterns,math}

% Beamer theme
\usetheme{metropolis}

% Title page info
\title{Probabilistic Algorithms: What, Why, and How}
\subtitle{A Deep Dive into Randomness in Computing}
\author{Sailesh Dahal}
\institute{Kathmandu University}
\date{\today}

% Bibliography
\addbibresource{refs.bib}

% Listings configuration
\lstset{
    language=Python,
    basicstyle=\ttfamily\small,
    breaklines=true,
    frame=single,
    numbers=left,
    numberstyle=\tiny,
    keywordstyle=\color{blue},
    commentstyle=\color{green!60!black},
    stringstyle=\color{red},
    showstringspaces=false
}


\begin{document}

\begin{frame}
  \begin{tikzpicture}[remember picture,overlay]
    \node[anchor=south east, xshift=-5mm, yshift=5mm] at (current page.south east) {
      \includegraphics[width=2cm]{../assets/ku_logo.png}
    };
  \end{tikzpicture}
  \titlepage
\end{frame}

% Outline frame
\begin{frame}{Outline}
  \tableofcontents
\end{frame}

% What are Probabilistic Algorithms?
\section{What are Probabilistic Algorithms?}
\begin{frame}{What is a Probabilistic Algorithm?}
  \begin{block}{Definition}
    An algorithm that makes random choices during execution to influence its behavior or output.
  \end{block}
\end{frame}

% Combined Visual: Deterministic vs Probabilistic Algorithm
\begin{frame}{Deterministic vs Probabilistic Algorithm}
  % Deterministic Algorithm (top)
  \begin{center}
    \begin{tikzpicture}
      % Box
      \node[draw, fill=yellow!80, minimum width=4cm, minimum height=1.5cm] (box) {\footnotesize Deterministic Algorithm};
      % Input arrow
      \node[left=1.5cm of box, align=center] (input) {\footnotesize Input $x$};
      \draw[->, thick] (input) -- (box);
      % Output arrow
      \node[right=1.5cm of box, align=center] (output) {\footnotesize Output $y$};
      \draw[->, thick] (box) -- (output);
    \end{tikzpicture}
    \\[0.2em]
    {\footnotesize Deterministic}
  \end{center}
  \pause
  \vspace{1.5em}
  % Probabilistic Algorithm (bottom)
  \begin{center}
    \begin{tikzpicture}
      % Box
      \node[draw, fill=gray!30, minimum width=4cm, minimum height=1.5cm] (box) {\footnotesize Randomized Algorithm};
      % Input arrow
      \node[left=1.5cm of box, align=center] (input) {\footnotesize Input $x$};
      \draw[->, thick] (input) -- (box);
      % Output arrow
      \node[right=1.5cm of box, align=center] (output) {\footnotesize Output $y_r$};
      \draw[->, thick] (box) -- (output);
      % Random bits arrow
      \node[above=0.8cm of box, align=center] (rand) {\footnotesize random bits $r$};
      \draw[->, thick] (rand) -- (box);
    \end{tikzpicture}
    \\[0.2em]
    {\footnotesize Probabilistic}
  \end{center}
\end{frame}

\begin{frame}{Types of Probabilistic Algorithms}
  \begin{itemize}
    \item Las Vegas Algorithms
    \item Monte Carlo Algorithms
  \end{itemize}
\end{frame}
% Expanded: Las Vegas Algorithms

\begin{frame}{Las Vegas Algorithms}
\begin{block}{Definition}
  Always produce a correct result, but the running time is a random variable (depends on random choices).
\end{block}
\pause
\begin{itemize}
\item Output is always correct (no error), but time may vary between runs.
\pause
\item Useful when correctness is critical, but we can tolerate variable performance.
\pause
\item \textbf{Example:} Randomized Quicksort
\begin{itemize}
  \item Randomly selects a pivot to avoid worst-case input.
  \item Always sorts correctly, but running time depends on random choices.
\end{itemize}
\pause
\item \textbf{Other Examples:}
\begin{itemize}
  \item Randomized algorithms for minimum cut in graphs (Karger's algorithm)
\end{itemize}
\pause
\item \textbf{Practical Usage in Software:}
\begin{itemize}
            \item Randomized Quicksort is widely used in standard libraries (e.g., C++ STL, Java, Python) for efficient sorting, especially to avoid worst-case performance on adversarial inputs.
            \item Karger's algorithm is used in network reliability analysis tools.
            \end{itemize}
            \end{itemize}
            \end{frame}

% Expanded: Monte Carlo Algorithms
\begin{frame}{Monte Carlo Algorithms}
\begin{block}{Definition}
  Always finish in a fixed amount of time, but may produce an incorrect result with a small probability.
\end{block}
\pause
\begin{itemize}
\item Running time is predictable (usually polynomial), but there is a chance of error.
\pause
\item Useful when speed is more important than absolute certainty, and errors are rare or can be tolerated.
\pause
\item \textbf{Example:} Primality Testing (Miller-Rabin)
\begin{itemize}
  \item Quickly determines if a number is prime, but may err with small probability.
  \item Error probability can be reduced by repeating the test.
\end{itemize}
\pause
\item \textbf{Other Examples:}
\begin{itemize}
  \item Monte Carlo integration (estimating $\pi$)
  \item Randomized algorithms for approximate counting
\end{itemize}
\pause
\item \textbf{Practical Usage in Software:}
\begin{itemize}
  \item Miller-Rabin primality test is used in cryptographic libraries (e.g., OpenSSL, GnuPG) for generating large prime numbers.
  \item Monte Carlo integration is used in computer graphics (e.g., rendering engines for global illumination), and in financial modeling for risk analysis.
\end{itemize}
  \end{itemize}
\end{frame}

% New section for applications
\section{Applications of Probabilistic Algorithms}

\begin{frame}{Real-World Motivation}
  \begin{itemize}
    \item Web search (PageRank)
    \item Load balancing (power of two choices)
    \item Hashing (universal hash functions)
    \item Primality testing (Miller-Rabin)
  \end{itemize}
\end{frame}

% Why use Probabilistic Algorithms?
\section{Why Probabilistic Algorithms?}
\begin{frame}{Why Randomness?}
  \begin{block}{Motivation}
    \begin{itemize}
      \item Simpler algorithms

      \item Better expected performance

      \item Avoid worst-case scenarios

      \item Useful for large-scale and distributed systems
    \end{itemize}
  \end{block}
\end{frame}

% How do Probabilistic Algorithms work?
\section{How do Probabilistic Algorithms Work?}
\begin{frame}{How: Randomization in Algorithms}
  \begin{block}{Key Idea}
    Use random choices to influence the algorithm's path or output.
  \end{block}

  \begin{itemize}
    \item Random pivot in Quicksort

    \item Random walks in graphs

    \item Random sampling
  \end{itemize}
\end{frame}

% Example: Randomized Quicksort
\section{Example: Randomized Quicksort}

% New slide: QuickSort vs Randomized QuickSort steps
\begin{frame}{QuickSort vs Randomized QuickSort}
  \textbf{QuickSort:}
  \begin{enumerate}
    \item Pick a pivot element from the array
    \item Split array into 3 subarrays: those smaller than pivot, those larger than pivot, and the pivot itself
    \item Recursively sort the subarrays, and concatenate them
  \end{enumerate}
  \vspace{1em}
  \textbf{Randomized QuickSort:}
  \begin{enumerate}
    \item Pick a pivot element \textbf{uniformly at random} from the array
    \item Split array into 3 subarrays: those smaller than pivot, those larger than pivot, and the pivot itself
    \item Recursively sort the subarrays, and concatenate them
  \end{enumerate}
\end{frame}

% New slide: Worst-case for QuickSort
\begin{frame}{Example: Randomized Quicksort}
  \textbf{Recall:} QuickSort can take $\Omega(n^2)$ time to sort an array of size $n$.
\end{frame}

% New slide: Theorem and expectation for Randomized QuickSort
\begin{frame}{Randomized QuickSort: Expected Runtime}
  \textbf{Theorem}
  \begin{block}{}
    Randomized QuickSort sorts a given array of length $n$ in $O(n \log n)$ expected time.
  \end{block}
  \vspace{1em}
  \textbf{Note:} On every input, randomized QuickSort takes $O(n \log n)$ time in expectation. On every input, it may take $\Omega(n^2)$ time with some small probability.
\end{frame}

\section{Example: Randomized Quicksort}

% New slide: QuickSort vs Randomized QuickSort steps
\begin{frame}{QuickSort vs Randomized QuickSort}
  \textbf{QuickSort:}
  \begin{enumerate}
    \item Pick a pivot element from the array \parencite{10.1093/comjnl/5.1.10}
    \item Split array into 3 subarrays: those smaller than pivot, those larger than pivot, and the pivot itself
    \item Recursively sort the subarrays, and concatenate them
  \end{enumerate}
  \vspace{1em}
  \textbf{Randomized QuickSort:}
  \begin{enumerate}
    \item Pick a pivot element \textbf{uniformly at random} from the array \parencite{motwani1995randomized}
    \item Split array into 3 subarrays: those smaller than pivot, those larger than pivot, and the pivot itself
    \item Recursively sort the subarrays, and concatenate them
  \end{enumerate}
\end{frame}

% New slide: Worst-case for QuickSort
\begin{frame}{Example: Randomized Quicksort}
  \textbf{Recall:} QuickSort can take $\Omega(n^2)$ time to sort an array of size $n$ \parencite{359631}
\end{frame}

% New slide: Theorem and expectation for Randomized QuickSort
\begin{frame}{Randomized QuickSort: Expected Runtime}
  \textbf{Theorem}
  \begin{block}{}
    Randomized QuickSort sorts a given array of length $n$ in $O(n \log n)$ expected time. \parencite{journals/acta/Sedgewick77}
  \end{block}
  \vspace{1em}
  \textbf{Note:} On every input, randomized QuickSort takes $O(n \log n)$ time in expectation. On every input, it may take $\Omega(n^2)$ time with some small probability.
\end{frame}

\subsection{Step-by-Step Execution}
\begin{frame}{Randomized Quicksort: Step 1 (Initial Array)}
  % Use columns for top-aligned split layout
  \begin{columns}[t]
    \column{0.6\textwidth}
    Consider the array:
    \[
      \renewcommand{\arraystretch}{1.5}
      \begin{array}{|c|c|c|c|c|c|c|c|}
        \hline
        15 & 3 & 1 & 10 & 9 & 0 & 6 & 4 \\
        \hline
      \end{array}
    \]
  \end{columns}
\end{frame}

\begin{frame}{Randomized Quicksort: Step 1.1 (Pivot Chosen)}
  % Use columns for top-aligned split layout
  \begin{columns}[t]
    \column{0.6\textwidth}
    Suppose the random pivot chosen is \textcolor{red}{10} (at index 3):
    \[
      \renewcommand{\arraystretch}{1.5}
      \begin{array}{|c|c|c|c|c|c|c|c|}
        \hline
        15 & 3 & 1 & \cellcolor{red!20}\textcolor{red}{10} & 9 & 0 & 6 & 4 \\
        \hline
      \end{array}
    \]

    \column{0.38\textwidth}
    \begin{minipage}[t]{\linewidth}
      \vspace{0pt}
      \begin{center}
        \begin{tikzpicture}[
          scale=0.85,
          transform shape,
          level distance=1.5cm,
          level 1/.style={sibling distance=3.0cm},
          level 2/.style={sibling distance=2.0cm},
          level 3/.style={sibling distance=1.2cm},
          every node/.style={font=\tiny}
        ]
          \node[circle, draw, fill=green!20, minimum size=1cm, align=center] (root) {
            \textbf{A[0,7]} \\
            10
          };
        \end{tikzpicture}
      \end{center}
    \end{minipage}
  \end{columns}
\end{frame}
% Step 2: Partitioning around 10
\begin{frame}{Randomized Quicksort: Step 2 (Partitioning Around Pivot 10)}
  % Use columns for top-aligned split layout
  \begin{columns}[t]
    \column{0.6\textwidth}
    After selecting pivot 10, we partition the array:
    \begin{itemize}
      \item \textcolor{green!60!black}{Left:} 4, 3, 1, 9, 0, 6 (elements before pivot position)
      \item \textcolor{red}{Middle:} 10 (pivot)
      \item \textcolor{blue}{Right:} 15 (element after pivot position)
    \end{itemize}

    After partitioning:
    \[
      \renewcommand{\arraystretch}{1.5}
      \begin{array}{|c|c|c|c|c|c|c|c|}
        \hline
        \cellcolor{green!20}\textcolor{green!60!black}{4} & \cellcolor{green!20}\textcolor{green!60!black}{3} & \cellcolor{green!20}\textcolor{green!60!black}{1} & \cellcolor{green!20}\textcolor{green!60!black}{9} & \cellcolor{green!20}\textcolor{green!60!black}{0} & \cellcolor{green!20}\textcolor{green!60!black}{6} & \cellcolor{red!20}\textcolor{red}{10} & 15 \\
        \hline
      \end{array}
    \]

    \column{0.38\textwidth}
    \begin{minipage}[t]{\linewidth}
      \vspace{0pt}
      \begin{center}
        \begin{tikzpicture}[
          scale=0.85,
          transform shape,
          level distance=1.5cm,
          level 1/.style={sibling distance=3.0cm},
          level 2/.style={sibling distance=2.0cm},
          level 3/.style={sibling distance=1.2cm},
          every node/.style={font=\tiny}
        ]
          % Root node
          \node[circle, draw, fill=green!20, minimum size=1cm, align=center] (root) {
            \textbf{A[0,7]} \\
            10
          }
          % Left child
          child {node[circle, draw, fill=green!20, minimum size=1cm, align=center] {
                  \textbf{A[0,5]}
                }}
          % Right child
          child {node[circle, draw, fill=white, minimum size=1cm, align=center] {
                  \textbf{A[7,7]}
                }};
        \end{tikzpicture}
      \end{center}
    \end{minipage}
  \end{columns}
\end{frame}

% Step 3: Recurse Left [A[0,5]], Pivot 4
\begin{frame}{Randomized Quicksort: Step 3 (Recurse Left [A[0,5]], Pivot 4)}
  % Use columns for top-aligned split layout
  \begin{columns}[t]
    \column{0.6\textwidth}
    Recurse on the left subarray:
    \\[0.5em]
    Let's choose a random pivot, say 4.
    \[
      \renewcommand{\arraystretch}{1.5}
      \begin{array}{|c|c|c|c|c|c|}
        \hline
        \cellcolor{red!20}\textcolor{red}{4} & 3 & 1 & 9 & 0 & 6 \\
        \hline
      \end{array}
    \]

    \column{0.38\textwidth}
    \begin{minipage}[t]{\linewidth}
      \vspace{0pt}
      \begin{center}
        \begin{tikzpicture}[
          scale=0.85,
          transform shape,
          level distance=1.5cm,
          level 1/.style={sibling distance=3.0cm},
          level 2/.style={sibling distance=2.0cm},
          level 3/.style={sibling distance=1.2cm},
          every node/.style={font=\tiny}
        ]
          % Root node
          \node[circle, draw, fill=green!20, minimum size=1cm, align=center] (root) {
            \textbf{A[0,7]} \\
            10
          }
          % Left child
          child {node[circle, draw, fill=green!20, minimum size=1cm, align=center] {
                  \textbf{A[0,5]} \\
                  4
                }}
          % Right child
          child {node[circle, draw, fill=white, minimum size=1cm, align=center] {
                  \textbf{A[7,7]}
                }};
        \end{tikzpicture}
      \end{center}
    \end{minipage}
  \end{columns}
\end{frame}

% Step 3.1: Partition Left [A[0,5]] Around 4
\begin{frame}{Randomized Quicksort: Step 3.1 (Partition Left [A[0,5]] Around 4)}
  % Use columns for top-aligned split layout
  \begin{columns}[t]
    \column{0.6\textwidth}
    After partitioning the left subarray:
    \[
      \renewcommand{\arraystretch}{1.5}
      \begin{array}{|c|c|c|c|c|c|}
        \hline
        \cellcolor{green!20}\textcolor{green!60!black}{0} & \cellcolor{green!20}\textcolor{green!60!black}{3} & \cellcolor{green!20}\textcolor{green!60!black}{1} & \cellcolor{red!20}\textcolor{red}{4} & 9 & 6 \\
        \hline
      \end{array}
    \]
    Partition:
    \begin{itemize}
      \item \textcolor{green!60!black}{Left:} 0, 3, 1 (elements before pivot)
      \item \textcolor{red}{Middle:} 4 (pivot)
      \item \textcolor{blue}{Right:} 9, 6 (elements after pivot)
    \end{itemize}

    \column{0.38\textwidth}
    \begin{minipage}[t]{\linewidth}
      \vspace{0pt}
      \begin{center}
        \begin{tikzpicture}[
          scale=0.85,
          transform shape,
          level distance=1.5cm,
          level 1/.style={sibling distance=3.0cm},
          level 2/.style={sibling distance=2.0cm},
          level 3/.style={sibling distance=1.2cm},
          every node/.style={font=\tiny}
        ]
          % Root node
          \node[circle, draw, fill=green!20, minimum size=1cm, align=center] (root) {
            \textbf{A[0,7]} \\
            10
          }
          % Left child
          child {node[circle, draw, fill=green!20, minimum size=1cm, align=center] {
                  \textbf{A[0,5]} \\
                  4
                }
              child {node[circle, draw, fill=white, minimum size=1cm, align=center] {
                      \textbf{A[0,2]}
                    }}
              child {node[circle, draw, fill=white, minimum size=1cm, align=center] {
                      \textbf{A[4,5]}
                    }}
            }
          % Right child
          child {node[circle, draw, fill=white, minimum size=1cm, align=center] {
                  \textbf{A[7,7]}
                }};
        \end{tikzpicture}
      \end{center}
    \end{minipage}
  \end{columns}
\end{frame}

% Step 3.1.1: Recurse Left [A[0,2]], Pivot 0
\begin{frame}{Randomized Quicksort: Step 3.1.1 (Recurse Left [A[0,2]], Pivot 0)}
  % Use columns for top-aligned split layout
  \begin{columns}[t]
    \column{0.6\textwidth}
    Recurse on the left subarray:
    \\[0.5em]
    Let's choose a random pivot, say 0.
    \[
      \renewcommand{\arraystretch}{1.5}
      \begin{array}{|c|c|c|}
        \hline
        \cellcolor{red!20}\textcolor{red}{0} & 3 & 1 \\
        \hline
      \end{array}
    \]

    \column{0.38\textwidth}
    \begin{minipage}[t]{\linewidth}
      \vspace{0pt}
      \begin{center}
        \begin{tikzpicture}[
          scale=0.85,
          transform shape,
          level distance=1.5cm,
          level 1/.style={sibling distance=3.0cm},
          level 2/.style={sibling distance=2.0cm},
          level 3/.style={sibling distance=1.2cm},
          every node/.style={font=\tiny}
        ]
          % Root node
          \node[circle, draw, fill=green!20, minimum size=1cm, align=center] (root) {
            \textbf{A[0,7]} \\
            10
          }
          % Left child
          child {node[circle, draw, fill=green!20, minimum size=1cm, align=center] {
                  \textbf{A[0,5]} \\
                  4
                }
              child {node[circle, draw, fill=green!20, minimum size=1cm, align=center] {
                      \textbf{A[0,2]}\\
                      0
                    }}
              child {node[circle, draw, fill=white, minimum size=1cm, align=center] {
                      \textbf{A[4,5]}
                    }}
            }
          % Right child
          child {node[circle, draw, fill=white, minimum size=1cm, align=center] {
                  \textbf{A[7,7]}
                }
            };
        \end{tikzpicture}
      \end{center}
    \end{minipage}
  \end{columns}
\end{frame}

% Step 3.1.1.1: Partition Left [A[0,2]] Around 0
\begin{frame}{Randomized Quicksort: Step 3.1.1.1 (Partition Left [A[0,2]] Around 0)}
  % Use columns for top-aligned split layout
  \begin{columns}[t]
    \column{0.6\textwidth}
    After partitioning the left subarray:
    \[
      \renewcommand{\arraystretch}{1.5}
      \begin{array}{|c|c|c|}
        \hline
        \cellcolor{red!20}\textcolor{red}{0} & 3 & 1 \\
        \hline
      \end{array}
    \]
    Partition:
    \begin{itemize}
      \item \textcolor{green!60!black}{Left:} (empty)
      \item \textcolor{red}{Middle:} 0 (pivot)
      \item \textcolor{blue}{Right:} 3, 1 (elements after pivot)
    \end{itemize}

    \column{0.38\textwidth}
    \begin{minipage}[t]{\linewidth}
      \vspace{0pt}
      \begin{center}
        \begin{tikzpicture}[
          scale=0.85,
          transform shape,
          level distance=1.5cm,
          level 1/.style={sibling distance=3.0cm},
          level 2/.style={sibling distance=2.0cm},
          level 3/.style={sibling distance=1.2cm},
          every node/.style={font=\tiny}
        ]
          % Root node
          \node[circle, draw, fill=green!20, minimum size=1cm, align=center] (root) {
            \textbf{A[0,7]} \\
            10
          }
          % Left child
          child {node[circle, draw, fill=green!20, minimum size=1cm, align=center] {
                  \textbf{A[0,5]} \\
                  4
                }
              child {node[circle, draw, fill=green!20, minimum size=1cm, align=center] {
                      \textbf{A[0,2]}\\
                      0
                    }
                  child {node[circle, draw, fill=red!40, minimum size=1cm, align=center] {
                          \textbf{A[0,-1]}
                        }}
                  child {node[circle, draw, fill=white, minimum size=1cm, align=center] {
                          \textbf{A[1,2]}
                        }}
                }
              child {node[circle, draw, fill=white, minimum size=1cm, align=center] {
                      \textbf{A[4,5]}
                    }}
            }
          % Right child
          child {node[circle, draw, fill=white, minimum size=1cm, align=center] {
                  \textbf{A[7,7]}
                }
            };
        \end{tikzpicture}
      \end{center}
    \end{minipage}
  \end{columns}
\end{frame}

\begin{frame}{Randomized Quicksort: Step 3.1.1.2 (Recurse Right [A[1,2]], Pivot 3)}
  % Use columns for top-aligned split layout
  \begin{columns}[t]
    \column{0.6\textwidth}
    Recurse on the right subarray:
    \\[0.5em]
    Let's choose a random pivot, say 3.
    \[
      \renewcommand{\arraystretch}{1.5}
      \begin{array}{|c|c|}
        \hline
        \cellcolor{red!20}\textcolor{red}{3} & 1 \\
        \hline
      \end{array}
    \]

    \column{0.38\textwidth}
    \begin{minipage}[t]{\linewidth}
      \vspace{0pt}
      \begin{center}
        \begin{tikzpicture}[
          scale=0.85,
          transform shape,
          level distance=1.5cm,
          level 1/.style={sibling distance=3.0cm},
          level 2/.style={sibling distance=2.0cm},
          level 3/.style={sibling distance=1.2cm},
          every node/.style={font=\tiny}
        ]
          % Root node
          \node[circle, draw, fill=green!20, minimum size=1cm, align=center] (root) {
            \textbf{A[0,7]} \\
            10
          }
          % Left child
          child {node[circle, draw, fill=green!20, minimum size=1cm, align=center] {
                  \textbf{A[0,5]} \\
                  4
                }
              child {node[circle, draw, fill=green!20, minimum size=1cm, align=center] {
                      \textbf{A[0,2]}\\
                      0
                    }
                  child {node[circle, draw, fill=red!40, minimum size=1cm, align=center] {
                          \textbf{A[0,-1]}
                        }}
                  child {node[circle, draw, fill=green!20, minimum size=1cm, align=center] {
                          \textbf{A[1,2]}\\
                          3
                        }
                    }
                }
              child {node[circle, draw, fill=white, minimum size=1cm, align=center] {
                      \textbf{A[4,5]}
                    }}
            }
          % Right child
          child {node[circle, draw, fill=white, minimum size=1cm, align=center] {
                  \textbf{A[7,7]}
                }
            };
        \end{tikzpicture}
      \end{center}
    \end{minipage}
  \end{columns}
\end{frame}

% Step 3.1.1.2.1: Partition [A[1,2]] Around 3
\begin{frame}{Randomized Quicksort: Step 3.1.1.2.1 (Partition [A[1,2]] Around 3)}
  % Use columns for top-aligned split layout
  \begin{columns}[t]
    \column{0.6\textwidth}
    After partitioning the left subarray:
    \[
      \renewcommand{\arraystretch}{1.5}
      \begin{array}{|c|c|c|}
        \hline
        1 & \cellcolor{red!20}\textcolor{red}{3} \\
        \hline
      \end{array}
    \]
    Partition:
    \begin{itemize}
      \item \textcolor{green!60!black}{Left:} 1 (element before pivot)
      \item \textcolor{red}{Middle:} 3 (pivot)
      \item \textcolor{blue}{Right:} (empty)
    \end{itemize}

    \column{0.38\textwidth}
    \begin{minipage}[t]{\linewidth}
      \vspace{0pt} % Ensures true top alignment
      \begin{center}

        \begin{tikzpicture}[
          scale=0.85,
          transform shape,
          level distance=1.5cm,
          level 1/.style={sibling distance=3.0cm},
          level 2/.style={sibling distance=2.0cm},
          level 3/.style={sibling distance=1.2cm},
          every node/.style={font=\tiny}
        ]
          % Root node
          \node[circle, draw, fill=green!20, minimum size=1cm, align=center] (root) {
            \textbf{A[0,7]} \\
            10
          }
          % Left child
          child {node[circle, draw, fill=green!20, minimum size=1cm, align=center] {
                  \textbf{A[0,5]} \\
                  4
                }
              child {node[circle, draw, fill=green!20, minimum size=1cm, align=center] {
                      \textbf{A[0,2]}\\
                      0
                    }
                  child {node[circle, draw, fill=red!40, minimum size=1cm, align=center] {
                          \textbf{A[0,-1]}
                        }}
                  child {node[circle, draw, fill=green!20, minimum size=1cm, align=center] {
                          \textbf{A[1,2]}\\
                          3
                        }
                      child {node[circle, draw, fill=blue!20, minimum size=1cm, align=center] {
                              \textbf{A[1,1]}
                            }
                        }
                      child {node[circle, draw, fill=white, minimum size=1cm, align=center] {
                              \textbf{A[3,2]}
                            }
                        }
                    }
                }
              child {node[circle, draw, fill=white, minimum size=1cm, align=center] {
                      \textbf{A[4,5]}
                    }}
            }
          % Right child
          child {node[circle, draw, fill=white, minimum size=1cm, align=center] {
                  \textbf{A[7,7]}
                }
            };
        \end{tikzpicture}
      \end{center}
    \end{minipage}
  \end{columns}
\end{frame}

% Step 3.1.1.2.1.1: Recurse Left [A[1,1]], Done
\begin{frame}{Randomized Quicksort: Step 3.1.1.2.1.1 (Recurse Left [A[1,1]], Done)}
  \begin{columns}[t]
    \column{0.6\textwidth}
    After partitioning the left subarray:
    \[
      \renewcommand{\arraystretch}{1.5}
      \begin{array}{|c|c|c|}
        \hline
        \cellcolor{green!20}{1} \\
        \hline
      \end{array}
    \]

    Partition:
    \begin{itemize}
      \item \textcolor{green!60!black}{Left:} (empty)
      \item \textcolor{red}{Middle:} 1 (pivot)
      \item \textcolor{blue}{Right:} (empty)
    \end{itemize}
    Single element subarray, done, return.
    \column{0.38\textwidth}
    \begin{minipage}[t]{\linewidth}
      \vspace{0pt} % Ensures true top alignment
      \begin{center}

        \begin{tikzpicture}[
          scale=0.85,
          transform shape,
          level distance=1.5cm,
          level 1/.style={sibling distance=3.0cm},
          level 2/.style={sibling distance=2.0cm},
          level 3/.style={sibling distance=1.2cm},
          every node/.style={font=\tiny}
        ]
          % Root node
          \node[circle, draw, fill=green!20, minimum size=1cm, align=center] (root) {
            \textbf{A[0,7]} \\
            10
          }
          % Left child
          child {node[circle, draw, fill=green!20, minimum size=1cm, align=center] {
                  \textbf{A[0,5]} \\
                  4
                }
              child {node[circle, draw, fill=green!20, minimum size=1cm, align=center] {
                      \textbf{A[0,2]}\\
                      0
                    }
                  child {node[circle, draw, fill=red!40, minimum size=1cm, align=center] {
                          \textbf{A[0,-1]}
                        }}
                  child {node[circle, draw, fill=green!20, minimum size=1cm, align=center] {
                          \textbf{A[1,2]}\\
                          3
                        }
                      child {node[circle, draw, fill=blue!20, minimum size=1cm, align=center] {
                              \textbf{A[1,1]}
                            }
                        }
                      child {node[circle, draw, fill=red!40, minimum size=1cm, align=center] {
                              \textbf{A[3,2]}
                            }
                        }
                    }
                }
              child {node[circle, draw, fill=white, minimum size=1cm, align=center] {
                      \textbf{A[4,5]}
                    }}
            }
          % Right child
          child {node[circle, draw, fill=white, minimum size=1cm, align=center] {
                  \textbf{A[7,7]}
                }
            };
        \end{tikzpicture}
      \end{center}
    \end{minipage}
  \end{columns}
\end{frame}

% Step 3.1.1.2.1.2: Recurse Right [A[3,2]], Done
\begin{frame}{Randomized Quicksort: Step 3.1.1.2.1.2 (Recurse Right [A[3,2]], Done)}
  \begin{columns}[t]
    \column{0.6\textwidth}
    Recurse on the right subarray $A[3,2]$ (empty, done).
    Return to parent call $A[1,2]$
    \column{0.38\textwidth}
    \begin{minipage}[t]{\linewidth}
      \vspace{0pt} % Ensures true top alignment
      \begin{center}

        \begin{tikzpicture}[
          scale=0.85,
          transform shape,
          level distance=1.5cm,
          level 1/.style={sibling distance=3.0cm},
          level 2/.style={sibling distance=2.0cm},
          level 3/.style={sibling distance=1.2cm},
          every node/.style={font=\tiny}
        ]
          % Root node
          \node[circle, draw, fill=green!20, minimum size=1cm, align=center] (root) {
            \textbf{A[0,7]} \\
            10
          }
          % Left child
          child {node[circle, draw, fill=green!20, minimum size=1cm, align=center] {
                  \textbf{A[0,5]} \\
                  4
                }
              child {node[circle, draw, fill=green!20, minimum size=1cm, align=center] {
                      \textbf{A[0,2]}\\
                      0
                    }
                  child {node[circle, draw, fill=red!40, minimum size=1cm, align=center] {
                          \textbf{A[0,-1]}
                        }}
                  child {node[circle, draw, fill=green!20, minimum size=1cm, align=center] {
                          \textbf{A[1,2]}\\
                          3
                        }
                      child {node[circle, draw, fill=blue!20, minimum size=1cm, align=center] {
                              \textbf{A[1,1]}
                            }
                        }
                      child {node[circle, draw, fill=red!40, minimum size=1cm, align=center] {
                              \textbf{A[3,2]}
                            }
                        }
                    }
                }
              child {node[circle, draw, fill=white, minimum size=1cm, align=center] {
                      \textbf{A[4,5]}
                    }}
            }
          % Right child
          child {node[circle, draw, fill=white, minimum size=1cm, align=center] {
                  \textbf{A[7,7]}
                }
            };
        \end{tikzpicture}
      \end{center}
    \end{minipage}
  \end{columns}
\end{frame}

% Step 3.1.2: Return to [A[0,2]]
\begin{frame}{Randomized Quicksort: Step 3.1.2 (Return to [A[0,2]])}
  \begin{columns}[t]
    \column{0.6\textwidth}
    Return to parent call $A[0,2]$
    \column{0.38\textwidth}
    \begin{minipage}[t]{\linewidth}
      \vspace{0pt} % Ensures true top alignment
      \begin{center}

        \begin{tikzpicture}[
          scale=0.85,
          transform shape,
          level distance=1.5cm,
          level 1/.style={sibling distance=3.0cm},
          level 2/.style={sibling distance=2.0cm},
          level 3/.style={sibling distance=1.2cm},
          every node/.style={font=\tiny}
        ]
          % Root node
          \node[circle, draw, fill=green!20, minimum size=1cm, align=center] (root) {
            \textbf{A[0,7]} \\
            10
          }
          % Left child
          child {node[circle, draw, fill=green!20, minimum size=1cm, align=center] {
                  \textbf{A[0,5]} \\
                  4
                }
              child {node[circle, draw, fill=green!20, minimum size=1cm, align=center] {
                      \textbf{A[0,2]}\\
                      0
                    }
                  child {node[circle, draw, fill=red!40, minimum size=1cm, align=center] {
                          \textbf{A[0,-1]}
                        }}
                  child {node[circle, draw, fill=blue!20, minimum size=1cm, align=center] {
                          \textbf{A[1,2]}\\
                          3
                        }
                      child {node[circle, draw, fill=blue!20, minimum size=1cm, align=center] {
                              \textbf{A[1,1]}\\
                              1
                            }
                        }
                      child {node[circle, draw, fill=red!40, minimum size=1cm, align=center] {
                              \textbf{A[3,2]}
                            }
                        }
                    }
                }
              child {node[circle, draw, fill=white, minimum size=1cm, align=center] {
                      \textbf{A[4,5]}
                    }}
            }
          % Right child
          child {node[circle, draw, fill=white, minimum size=1cm, align=center] {
                  \textbf{A[7,7]}
                }
            };
        \end{tikzpicture}
      \end{center}
    \end{minipage}
  \end{columns}
\end{frame}

% Step 3.2: Recurse Right [A[4,5]], Pivot 9
\begin{frame}{Randomized Quicksort: Step 3.2 (Recurse Right [A[4,5]], Pivot 9)}
  \begin{columns}[t]
    \column{0.6\textwidth}
    Recurse on the right subarray $A[4,5]$:
    \\[0.5em]
    Let's choose a random pivot, say 9.
    \[
      \renewcommand{\arraystretch}{1.5}
      \begin{array}{|c|c|}
        \hline
        \cellcolor{red!20}\textcolor{red}{9} & 6 \\
        \hline
      \end{array}
    \]

    \column{0.38\textwidth}
    \begin{minipage}[t]{\linewidth}
      \vspace{0pt} % Ensures true top alignment
      \begin{center}

        \begin{tikzpicture}[
          scale=0.85,
          transform shape,
          level distance=1.5cm,
          level 1/.style={sibling distance=3.0cm},
          level 2/.style={sibling distance=2.0cm},
          level 3/.style={sibling distance=1.2cm},
          every node/.style={font=\tiny}
        ]
          % Root node
          \node[circle, draw, fill=green!20, minimum size=1cm, align=center] (root) {
            \textbf{A[0,7]} \\
            10
          }
          % Left child
          child {node[circle, draw, fill=green!20, minimum size=1cm, align=center] {
                  \textbf{A[0,5]} \\
                  4
                }
              child {node[circle, draw, fill=green!20, minimum size=1cm, align=center] {
                      \textbf{A[0,2]}\\
                      0
                    }
                  child {node[circle, draw, fill=red!40, minimum size=1cm, align=center] {
                          \textbf{A[0,-1]}
                        }}
                  child {node[circle, draw, fill=blue!20, minimum size=1cm, align=center] {
                          \textbf{A[1,2]}\\
                          3
                        }
                      child {node[circle, draw, fill=blue!20, minimum size=1cm, align=center] {
                              \textbf{A[1,1]}\\
                              1
                            }
                        }
                      child {node[circle, draw, fill=red!40, minimum size=1cm, align=center] {
                              \textbf{A[3,2]}
                            }
                        }
                    }
                }
              child {node[circle, draw, fill=green!20, minimum size=1cm, align=center] {
                      \textbf{A[4,5]}\\
                      9
                    }
                    child {node[circle, draw, fill=white, minimum size=1cm, align=center] {
                    \textbf{A[4,4]}
                        }}
                  child {node[circle, draw, fill=white, minimum size=1cm, align=center] {
                      \textbf{A[6,5]}
                    }}
                  }
            }
          % Right child
          child {node[circle, draw, fill=white, minimum size=1cm, align=center] {
                  \textbf{A[7,7]}
                }
            };
        \end{tikzpicture}
      \end{center}
    \end{minipage}
  \end{columns}
\end{frame}

% Step 3.2.1: Partition [A[4,5]] Around 9
\begin{frame}{Randomized Quicksort: Step 3.2.1 (Partition [A[4,5]] Around 9)}
  \begin{columns}[t]
    \column{0.6\textwidth}
    Recurse on the right subarray $A[4,5]$:
    \\[0.5em]

    Suppose the random pivot is \textcolor{red}{9}:
    \[
      \renewcommand{\arraystretch}{1.5}
      \begin{array}{|c|c|}
        \hline
        6 & \cellcolor{red!20}\textcolor{red}{9} \\
        \hline
      \end{array}
    \]

    Partition:
    \begin{itemize}
      \item \textcolor{green!60!black}{Left:} 6 (element before pivot)
      \item \textcolor{red}{Middle:} 9 (pivot)
      \item \textcolor{blue}{Right:} (empty)
    \end{itemize}

    \column{0.38\textwidth}
    \begin{minipage}[t]{\linewidth}
      \vspace{0pt} % Ensures true top alignment
      \begin{center}

        \begin{tikzpicture}[
          scale=0.85,
          transform shape,
          level distance=1.5cm,
          level 1/.style={sibling distance=3.0cm},
          level 2/.style={sibling distance=2.0cm},
          level 3/.style={sibling distance=1.2cm},
          every node/.style={font=\tiny}
        ]
          % Root node
          \node[circle, draw, fill=green!20, minimum size=1cm, align=center] (root) {
            \textbf{A[0,7]} \\
            10
          }
          % Left child
          child {node[circle, draw, fill=green!20, minimum size=1cm, align=center] {
                  \textbf{A[0,5]} \\
                  4
                }
              child {node[circle, draw, fill=green!20, minimum size=1cm, align=center] {
                      \textbf{A[0,2]}\\
                      0
                    }
                  child {node[circle, draw, fill=red!40, minimum size=1cm, align=center] {
                          \textbf{A[0,-1]}
                        }}
                  child {node[circle, draw, fill=blue!20, minimum size=1cm, align=center] {
                          \textbf{A[1,2]}\\
                          3
                        }
                      child {node[circle, draw, fill=blue!20, minimum size=1cm, align=center] {
                              \textbf{A[1,1]}\\
                              1
                            }
                        }
                      child {node[circle, draw, fill=red!40, minimum size=1cm, align=center] {
                              \textbf{A[3,2]}
                            }
                        }
                    }
                }
              child {node[circle, draw, fill=green!20, minimum size=1cm, align=center] {
                      \textbf{A[4,5]}\\
                      9
                    }
                  child {node[circle, draw, fill=white, minimum size=1cm, align=center] {
                          \textbf{A[4,4]}
                        }}
                  child {node[circle, draw, fill=white, minimum size=1cm, align=center] {
                          \textbf{A[6,5]}
                        }}
                }
            }
          % Right child
          child {node[circle, draw, fill=white, minimum size=1cm, align=center] {
                  \textbf{A[7,7]}
                }
            };
        \end{tikzpicture}
      \end{center}
    \end{minipage}
  \end{columns}
\end{frame}

% Step 3.2.1.1: Recurse Left [A[4,4]], Done
\begin{frame}{Randomized Quicksort: Step 3.2.1.1 (Recurse Left [A[4,4]], Done)}
  \begin{columns}[t]
    \column{0.6\textwidth}
    After partitioning the left subarray:
    \[
      \renewcommand{\arraystretch}{1.5}
      \begin{array}{|c|c|c|}
        \hline
        \cellcolor{green!20}{6} \\
        \hline
      \end{array}
    \]

    Partition:
    \begin{itemize}
      \item \textcolor{green!60!black}{Left:} (empty)
      \item \textcolor{red}{Middle:} 6 (pivot)
      \item \textcolor{blue}{Right:} (empty)
    \end{itemize}
    Single element subarray, done, return.
    \column{0.38\textwidth}
    \begin{minipage}[t]{\linewidth}
      \vspace{0pt} % Ensures true top alignment
      \begin{center}

        \begin{tikzpicture}[
          scale=0.85,
          transform shape,
          level distance=1.5cm,
          level 1/.style={sibling distance=3.0cm},
          level 2/.style={sibling distance=2.0cm},
          level 3/.style={sibling distance=1.2cm},
          every node/.style={font=\tiny}
        ]
          % Root node
          \node[circle, draw, fill=green!20, minimum size=1cm, align=center] (root) {
            \textbf{A[0,7]} \\
            10
          }
          % Left child
          child {node[circle, draw, fill=green!20, minimum size=1cm, align=center] {
                  \textbf{A[0,5]} \\
                  4
                }
              child {node[circle, draw, fill=green!20, minimum size=1cm, align=center] {
                      \textbf{A[0,2]}\\
                      0
                    }
                  child {node[circle, draw, fill=red!40, minimum size=1cm, align=center] {
                          \textbf{A[0,-1]}
                        }}
                  child {node[circle, draw, fill=blue!20, minimum size=1cm, align=center] {
                          \textbf{A[1,2]}\\
                          3
                        }
                      child {node[circle, draw, fill=blue!20, minimum size=1cm, align=center] {
                              \textbf{A[1,1]}\\
                              1
                            }
                        }
                      child {node[circle, draw, fill=red!40, minimum size=1cm, align=center] {
                              \textbf{A[3,2]}
                            }
                        }
                    }
                }
              child {node[circle, draw, fill=green!20, minimum size=1cm, align=center] {
                      \textbf{A[4,5]}
                      }}
            }
          % Right child
          child {node[circle, draw, fill=white, minimum size=1cm, align=center] {
                  \textbf{A[7,7]}
                }
            };
        \end{tikzpicture}
      \end{center}
    \end{minipage}
  \end{columns}
\end{frame}

% Step 3.2.1.2: Recurse Right [A[6,5]], Done
\begin{frame}{Randomized Quicksort: Step 3.2.1.2 (Recurse Right [A[6,5]], Done)}
  \begin{columns}[t]
    \column{0.6\textwidth}
    Recurse on the right subarray $A[6,5]$ (empty, done).
    Return to parent call $A[4,5]$\\
    Return to parent call $A[0,5]$\\
    \column{0.38\textwidth}
    \begin{minipage}[t]{\linewidth}
      \vspace{0pt} % Ensures true top alignment
      \begin{center}

        \begin{tikzpicture}[
          scale=0.85,
          transform shape,
          level distance=1.5cm,
          level 1/.style={sibling distance=3.0cm},
          level 2/.style={sibling distance=2.0cm},
          level 3/.style={sibling distance=1.2cm},
          every node/.style={font=\tiny}
        ]
          % Root node
          \node[circle, draw, fill=green!20, minimum size=1cm, align=center] (root) {
            \textbf{A[0,7]} \\
            10
          }
          % Left child
          child {node[circle, draw, fill=blue!20, minimum size=1cm, align=center] {
                  \textbf{A[0,5]} \\
                  4
                }
              child {node[circle, draw, fill=green!20, minimum size=1cm, align=center] {
                      \textbf{A[0,2]}\\
                      0
                    }
                  child {node[circle, draw, fill=red!40, minimum size=1cm, align=center] {
                          \textbf{A[0,-1]}
                        }}
                  child {node[circle, draw, fill=blue!20, minimum size=1cm, align=center] {
                          \textbf{A[1,2]}\\
                          3
                        }
                      child {node[circle, draw, fill=blue!20, minimum size=1cm, align=center] {
                              \textbf{A[1,1]}\\
                              1
                            }
                        }
                      child {node[circle, draw, fill=red!40, minimum size=1cm, align=center] {
                              \textbf{A[3,2]}
                            }
                        }
                    }
                }
              child {node[circle, draw, fill=blue!20, minimum size=1cm, align=center] {
                      \textbf{A[4,5]}\\
                      9
                    }
                  child {node[circle, draw, fill=blue!20, minimum size=1cm, align=center] {
                          \textbf{A[4,4]}
                        }}
                  child {node[circle, draw, fill=red!40, minimum size=1cm, align=center] {
                          \textbf{A[6,5]}
                        }}
                }
            }
          % Right child
          child {node[circle, draw, fill=white, minimum size=1cm, align=center] {
                  \textbf{A[7,7]}
                }
            };
        \end{tikzpicture}
      \end{center}
    \end{minipage}
  \end{columns}
\end{frame}

% Step 3.3: Recurse Right [A[7,7]], Done
\begin{frame}{Randomized Quicksort: Step 3.3 (Recurse Right [A[7,7]], Done)}
  \begin{columns}[t]
    \column{0.6\textwidth}
    Recurse on the right subarray $A[7,7]$
    After partitioning the right subarray:
    \[
      \renewcommand{\arraystretch}{1.5}
      \begin{array}{|c|c|c|}
        \hline
        \cellcolor{green!20}{1} \\
        \hline
      \end{array}
    \]

    Partition:
    \begin{itemize}
      \item \textcolor{green!60!black}{Left:} (empty)
      \item \textcolor{red}{Middle:} 1 (pivot)
      \item \textcolor{blue}{Right:} (empty)
    \end{itemize}
    Single element subarray, done, return.
    \column{0.38\textwidth}
    \begin{minipage}[t]{\linewidth}
      \vspace{0pt} % Ensures true top alignment
      \begin{center}

        \begin{tikzpicture}[
          scale=0.85,
          transform shape,
          level distance=1.5cm,
          level 1/.style={sibling distance=3.0cm},
          level 2/.style={sibling distance=2.0cm},
          level 3/.style={sibling distance=1.2cm},
          every node/.style={font=\tiny}
        ]
          % Root node
          \node[circle, draw, fill=green!20, minimum size=1cm, align=center] (root) {
            \textbf{A[0,7]} \\
            10
          }
          % Left child
          child {node[circle, draw, fill=blue!20, minimum size=1cm, align=center] {
                  \textbf{A[0,5]} \\
                  4
                }
              child {node[circle, draw, fill=green!20, minimum size=1cm, align=center] {
                      \textbf{A[0,2]}\\
                      0
                    }
                  child {node[circle, draw, fill=red!40, minimum size=1cm, align=center] {
                          \textbf{A[0,-1]}
                        }}
                  child {node[circle, draw, fill=blue!20, minimum size=1cm, align=center] {
                          \textbf{A[1,2]}\\
                          3
                        }
                      child {node[circle, draw, fill=blue!20, minimum size=1cm, align=center] {
                              \textbf{A[1,1]}\\
                              1
                            }
                        }
                      child {node[circle, draw, fill=red!40, minimum size=1cm, align=center] {
                              \textbf{A[3,2]}
                            }
                        }
                    }
                }
              child {node[circle, draw, fill=blue!20, minimum size=1cm, align=center] {
                      \textbf{A[4,5]}\\
                      9
                    }
                  child {node[circle, draw, fill=blue!20, minimum size=1cm, align=center] {
                          \textbf{A[4,4]}
                        }}
                  child {node[circle, draw, fill=red!40, minimum size=1cm, align=center] {
                          \textbf{A[6,5]}
                        }}
                }
            }
          % Right child
          child {node[circle, draw, fill=blue!20, minimum size=1cm, align=center] {
                  \textbf{A[7,7]} \\
                  15
                }
            };
        \end{tikzpicture}
      \end{center}
    \end{minipage}
  \end{columns}
\end{frame}

% Step 4: Final Sorted Array
\begin{frame}{Randomized Quicksort: Step 4 (Final Sorted Array)}
  The final sorted array is:
  \[
    \renewcommand{\arraystretch}{1.5}
    \begin{array}{|c|c|c|c|c|c|c|c|}
      \hline
      0 & 1 & 3 & 4 & 6 & 9 & 10 & 15 \\
      \hline
    \end{array}
  \]
\end{frame}

\begin{frame}{Quicksort Time Complexity}
  \begin{itemize}
    \item \textbf{Worst-case:} $O(n^2)$
    \item \textbf{Best-case:} $O(n \log n)$
    \item \textbf{Expected:} $O(n \log n)$ (randomized)
  \end{itemize}
\end{frame}


\section{Time Complexity Analysis of Randomized Quicksort}
% ./programs/analysis_quicksort/sorting_algorithms_benchmark.tex
% create a full frame image for this ignore the frame padding pages
\begin{frame}{Sorting Algorithms Benchmark}
  \begin{figure}[h]
    \centering
    \includegraphics[height=0.95\textheight]{./programs/analysis_quicksort/sorting_algorithms_benchmark.png}
  \end{figure}
\end{frame}

\begin{frame}{Quicksort Recurrence}
  \begin{block}{Expected Comparisons}
    Let $T(n)$ be the expected number of comparisons to sort $n$ distinct elements using randomized quicksort:
    \[
      T(n) \leq n + \frac{1}{n} \sum_{i=1}^n (T(i-1) + T(n-i))
    \]
  \end{block}

  \begin{itemize}
    \item $n$ comparisons in partitioning: each element compared to the pivot.
    \item Pivot is chosen uniformly at random.
    \item For pivot at position $i$, recursive calls sort subarrays of size $i - 1$ and $n - i$.
    \item We average over all $n$ possible pivot positions.
  \end{itemize}

  \begin{block}{Base Case}
    \[
      T(1) = 0 \quad \text{(single element requires no comparisons)}
    \]
  \end{block}

  \begin{block}{Asymptotic Solution}
    Solving the recurrence gives:
    \[
      T(n) = O(n \log n)
    \]
    \parencite{journals/acta/Sedgewick77}
  \end{block}
\end{frame}


\begin{frame}{Solving the Recurrence Step-by-Step}
  \textbf{Step 1: Write the Recurrence}
  \[
    T(n) \leq n + \frac{1}{n} \sum_{i=1}^n (T(i-1) + T(n-i))
  \]
  \textbf{Step 2: Guess the Solution}
  Assume $T(n) \leq cn \log n$ for some constant $c$.

\end{frame}
\begin{frame}{Solving the Recurrence Step-by-Step}
  \textbf{Step 3: Plug the Guess}
  \[
    T(n) \leq n + \frac{2c}{n} \sum_{k=1}^{n-1} k \log k
  \]

  Use:
  \[
    \sum_{k=1}^{n-1} k \log k \leq \frac{1}{2}n^2 \log n - \frac{1}{8}n^2
  \]

  Then:
  \[
    T(n) \leq n + c n \log n
  \]

  \textbf{Step 4: Conclusion}
  \[
    T(n) = O(n \log n)
  \]
\end{frame}

\section[Example: Monte Carlo Estimation of pi]{Example: Monte Carlo Estimation of π}
\begin{frame}{Monte Carlo Method - Estimating $\pi$}
  The Monte Carlo method \parencite{Metropolis01091949} estimates $\pi$ by simulating random points in a unit square and counting how many fall inside a quarter circle of radius 1. The ratio of points inside the circle to the total points, multiplied by 4, approximates $\pi$ \parencite{beckmann1971history}.
\end{frame}



\begin{frame}{Monte Carlo Algorithm}
  \begin{enumerate}
    \item Generate $N$ random points $(x, y)$ where $0 \leq x \leq 1$ and $0 \leq y \leq 1$.
    \item For each point, check if it lies inside the quarter circle: $x^2 + y^2 \leq 1$.
    \item Count the number of points $M$ that satisfy the condition.
    \item Estimate $\pi$ as: $\pi \approx 4 \times \frac{M}{N}$ \parencite{hammersley1964monte}.
  \end{enumerate}
\end{frame}

\begin{frame}{Visual Illustration}
  \begin{center}
    \begin{tikzpicture}[scale=3.5]
      % Draw square and quarter circle
      \draw[thick] (0,0) rectangle (1,1);
      \draw[thin, blue] (0,1) arc (90:0:1);

      \foreach \i in {1,...,300} {
          \pgfmathsetmacro{\x}{rnd}
          \pgfmathsetmacro{\y}{rnd}
          \pgfmathsetmacro{\r}{\x*\x + \y*\y}

          \ifdim\r pt<1pt
            \fill[green!70!black] (\x,\y) circle (0.005);
          \else
            \fill[red!70!black] (\x,\y) circle (0.005);
          \fi
        }

      % Labels
      \node[anchor=north] at (0.5,0) {Unit Square};
      \node[blue, anchor=west] at (0.6,0.3) {Quarter Circle};
    \end{tikzpicture}
  \end{center}
\end{frame}


\begin{frame}{Example Calculation}
  \begin{itemize}
    \item Suppose we generate $N = 1000$ random points in the unit square
    \item After simulation, we count $M = 785$ points inside the quarter circle
    \item We estimate $\pi$ as:
          \[ \pi \approx 4 \times \frac{M}{N} = 4 \times \frac{785}{1000} = 3.14 \]
    \item The true value of $\pi$ is approximately $3.14159$ \parencite{beckmann1971history}
  \end{itemize}
\end{frame}

\begin{frame}{Convergence and Error Analysis}
  \begin{itemize}
    \item The error in our estimate decreases as $O(1/\sqrt{N})$ \parencite{kalos2008monte}
    \item This means:
          \begin{itemize}
            \item $N=100$ points gives roughly 10\% error
            \item $N=10,000$ points gives roughly 1\% error
            \item $N=1,000,000$ points gives roughly 0.1\% error
          \end{itemize}
    \item The Monte Carlo method is especially useful for calculating multidimensional integrals \parencite{MonteCarloCookson2005}
    \item For $\pi$ calculation, there are more efficient methods, but this one is visually intuitive
  \end{itemize}
\end{frame}

\begin{frame}{Demo Visualization}
  \begin{center}
    \includegraphics[width=0.4\textwidth]{./programs/pi_calculation/pi_calculation.png}

    \vspace{0.5cm}
    \href{./programs/pi_calculation/pi_calculation.html}{Open Interactive Demo}
    \end{center}
\end{frame}


\section{Probabilistic Data Structures}
\begin{frame}{Deterministic vs. Probabilistic Data Structures}
  \begin{columns}
    \begin{column}{0.5\textwidth}
      \textbf{Deterministic (e.g., Hash Set, List):}
      \begin{itemize}
        \item Always provide exact answers.
        \item Can be space-intensive (store all elements).
        \item Operations might be slower for large datasets (e.g., disk I/O).
        \item \textbf{Guarantee:} No errors (false positives or negatives).
      \end{itemize}
    \end{column}
    \begin{column}{0.5\textwidth}
      \textbf{Probabilistic (e.g., Bloom Filter):}
      \begin{itemize}
        \item Provide approximate answers with controlled error.
        \item Very space-efficient (use bits, not full elements).
        \item Operations are typically very fast (constant time).
        \item \textbf{Trade-off:} Small error probability for huge efficiency gains.
      \end{itemize}
    \end{column}
  \end{columns}

  \begin{block}{Key Idea}
    Use PDS when approximate answers are acceptable and space/speed are critical.
  \end{block}
\end{frame}

\begin{frame}{Example: Why PDS? Username Availability}
  \begin{block}{The Problem}
    A website with millions of users needs to instantly check if a username is available during registration. How? \parencite{10.1145/362686.362692}
  \end{block}

  \begin{columns}
    \begin{column}{0.5\textwidth}
      \textbf{Deterministic Approach (Database Query):}
      \begin{itemize}
        \item Store all usernames in a database.
        \item Query DB: `SELECT 1 FROM users WHERE username = ?`
        \item \textbf{Accurate? Yes.}
        \item \textbf{Fast? No.} Requires disk I/O, network latency.
        \item \textbf{Scalable? Poorly.} High load on DB servers.
      \end{itemize}
    \end{column}
    \begin{column}{0.5\textwidth}
      \textbf{Probabilistic Approach (Bloom Filter):}
      \begin{itemize}
        \item Keep a compact Bloom filter in memory \parencite{10.1145/362686.362692}.
        \item Check filter: Is `username` possibly present?
        \item \textbf{Accurate? Mostly.} Small chance of false positive (saying taken when available), needs DB check then.
        \item \textbf{Fast? Yes.} In-memory check is O(k).
        \item \textbf{Scalable? Excellently.} Drastically reduces DB load.
      \end{itemize}
    \end{column}
  \end{columns}
\end{frame}

\begin{frame}{Username Checking: Implementation Details}
  \begin{enumerate}
    \item \textbf{Initialization:} Load all existing usernames into Bloom filter at service startup (only infrequent DB reads).
    \item \textbf{New registrations:} Add username to both database and Bloom filter.
    \item \textbf{Availability check process:}
          \begin{itemize}
            \item Check username against Bloom filter first (Fast, in-memory)
            \item If Bloom filter says "definitely not in set" $\rightarrow$ Username is available (99\% case for 1\% error rate)
            \item If Bloom filter says "possibly in set" $\rightarrow$ Verify with database query (Slow, but rare)
          \end{itemize}
  \end{enumerate}

  \begin{block}{Performance Impact (10M users, 1\% error)}
    \begin{itemize}
      \item Memory: $\approx 18$ MB Bloom Filter vs. hundreds of MB for DB index/cache.
      \item Speed: 99\% of availability checks avoid slow database queries. \parencite{UsernameProblem2012}
    \end{itemize}
  \end{block}
\end{frame}


\begin{frame}{Username Checking: System Architecture}
  \begin{center}
    \begin{tikzpicture}[
        block/.style={rectangle, draw, text width=2cm, text centered, minimum height=1cm},
        line/.style={draw, -latex},
        cloud/.style={draw, ellipse, minimum width=2cm, minimum height=1cm}
      ]

      % Client and servers
      \node[block] (client) at (0,0) {Client};
      \node[block] (api) at (4,0) {API Server};
      \node[block] (bloom) at (8,1) {Bloom Filter (Memory)};
      \node[block] (db) at (8,-1) {Database (Disk)};

      % Connections
      \path[line] (client) -- node[above] {Check username} (api);
      \path[line] (api) -- node[above] {1. Check Filter} (bloom);
      \path[line] (api) -- node[below] {2. Verify if needed} (db);

      % Fast path annotation
      \draw[dashed, thick, green!60!black, ->] (bloom) to[out=135, in=45] node[above] {Fast 'Available' (99\%)} (api);

      % Load path
      \path[line, gray] (db) -- node[right] {Initial load} (bloom);
    \end{tikzpicture}
  \end{center}

  \begin{itemize}
    \item Bloom filter acts as a fast, efficient preliminary check.
    \item Deterministic check (DB) used only as a fallback.
    \item Massively reduces load on the expensive resource (Database).
  \end{itemize}
\end{frame}

\begin{frame}{Bloom Filters: The Theory}
  \begin{columns}
    \begin{column}{0.6\textwidth}
      \begin{itemize}
        \item Space-efficient probabilistic data structure \parencite{10.1145/362686.362692}
        \item Tests if an element is a member of a set
        \item Possible false positives, never false negatives
        \item Components:
              \begin{itemize}
                \item Bit array of $m$ bits (initially all 0)
                \item $k$ independent hash functions
              \end{itemize}
      \end{itemize}
    \end{column}
    \begin{column}{0.4\textwidth}
      \begin{tikzpicture}[scale=0.5]
        % Draw bit array
        \foreach \i in {0,...,7} {
            \draw (\i,0) rectangle (\i+1,1);
            \node at (\i+0.5,0.5) {0};
          }

        % Show set bits
        \node at (1.5,0.5) {\textcolor{red}{1}};
        \node at (4.5,0.5) {\textcolor{red}{1}};
        \node at (6.5,0.5) {\textcolor{red}{1}};
      \end{tikzpicture}
    \end{column}
  \end{columns}
\end{frame}

\begin{frame}{Bloom Filter Operations}
  \begin{columns}
    \begin{column}{0.5\textwidth}
      \textbf{Add element:}
      \begin{enumerate}
        \item Hash element with $k$ functions
        \item Set bits at these $k$ positions to 1
      \end{enumerate}

      \textbf{Query element:}
      \begin{enumerate}
        \item Hash element with $k$ functions
        \item Check bits at these $k$ positions
        \item If \textbf{any} bit is 0: \textcolor{red}{Definitely not in set}
        \item If \textbf{all} bits are 1: \textcolor{orange}{Probably in set}
      \end{enumerate}
    \end{column}
    \begin{column}{0.5\textwidth}
      \begin{tikzpicture}[scale=0.5]
        % Draw bit array
        \foreach \i in {0,...,7} {
            \draw (\i,0) rectangle (\i+1,1);
            \node at (\i+0.5,0.5) {0};
          }

        % Update for apple
        \node at (1.5,0.5) {\textcolor{red}{1}};
        \node at (4.5,0.5) {\textcolor{red}{1}};
        \node at (6.5,0.5) {\textcolor{red}{1}};

        % Draw element and hash functions
        \node at (4,-1.5) {Query: "apple"};
        \draw[->] (4,-1.2) -- (1.5,0);
        \draw[->] (4,-1.2) -- (4.5,0);
        \draw[->] (4,-1.2) -- (6.5,0);
      \end{tikzpicture}
    \end{column}
  \end{columns}
\end{frame}

\begin{frame}{The Math Behind Bloom Filters}
  \begin{itemize}
    \item \textbf{False positive probability ($p$):}
          \begin{equation}
            p \approx \left(1 - e^{-kn/m}\right)^k
          \end{equation}\parencite{im/1109191032}

    \item \textbf{Optimal size ($m$ bits) for $n$ items, error $p$:}
          \begin{equation}
            m = -\frac{n \ln p}{(\ln 2)^2}
          \end{equation}

    \item \textbf{Optimal hash functions ($k$):}
          \begin{equation}
            k = \frac{m}{n} \ln 2 \approx 0.7 \cdot \frac{m}{n}
          \end{equation}
  \end{itemize}
\end{frame}

\begin{frame}{Time and Space Complexity}
  \begin{center}
    \begin{tabular}{l|c|c|c|c}
      \textbf{Structure} & \textbf{Space} & \textbf{Lookup} & \textbf{Insert} & \textbf{Error Type} \\
      \hline
      Hash Set           & $O(n)$         & $O(1)$ avg      & $O(1)$ avg      & None                \\
      Bloom Filter       & $O(m)$         & $O(k)$          & $O(k)$          & False Positives     \\
      Sorted List        & $O(n)$         & $O(\log n)$     & $O(n)$          & None                \\
      Trie               & $O(N)$         & $O(L)$          & $O(L)$          & None                \\
    \end{tabular}
  \end{center}

  $n$=items, $m$=bits ($m \ll n \times item\_size$), $k$=hashes, $N$=total chars, $L$=key length
\end{frame}

\begin{frame}{Bloom Filter Simulation}
  \begin{figure}[h]
    \centering
    \includegraphics[height=0.95\textheight]{./programs/bloom_filter_sim/bloom_filter_sim.png}
  \end{figure}
\end{frame}

\begin{frame}{Other Applications of Bloom Filters}
  \vspace{0.2cm}
  \begin{columns}
    \begin{column}{0.5\textwidth}
      \textbf{Web/Database:}
      \begin{itemize}
        \item Cache hit/miss optimization (e.g., CDNs)
        \item Avoid unnecessary DB lookups (like username example)
        \item Recommendation systems (seen items) \parencite{im/1109191032}
      \end{itemize}

      \textbf{Network:}
      \begin{itemize}
        \item Web crawler URL deduplication (avoid re-crawling)
        \item Network packet routing (track flows efficiently)
        \item P2P network resource discovery
      \end{itemize}
    \end{column}
    \begin{column}{0.5\textwidth}
      \textbf{Security:}
      \begin{itemize}
        \item Malware signature detection
        \item Spam filtering (known bad IPs/domains)
        \item Password breach checking (HaveIBeenPwned)
      \end{itemize}

      \textbf{Big Data:}
      \begin{itemize}
        \item Stream deduplication (unique visitors/events)
        \item Distributed data sync (approximate differences)
        \item Genomics (k-mer counting)
      \end{itemize}
    \end{column}
  \end{columns}
\end{frame}

\begin{frame}{When to Use Bloom Filters}
  Bloom filters are ideal when:

  \begin{itemize}
    \item Memory is a critical constraint (Big Data, embedded systems)
    \item False positives are acceptable (can be handled by a secondary check)
    \item False negatives are unacceptable (must find all true positives)
    \item Elements are expensive to store or compare
    \item Lookup speed is crucial (real-time systems)
    \item Deletions are not needed (or use variants like Counting Bloom Filters)
  \end{itemize}
\end{frame}

% Visual: Deterministic Algorithm


% References
\begin{frame}[allowframebreaks]{References}
  \printbibliography
\end{frame}

\end{document}
